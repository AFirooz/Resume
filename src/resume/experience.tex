%-------------------------------------------------------------------------------
%	SECTION TITLE
%-------------------------------------------------------------------------------

\cvsection{Experience}
\vspace*{-0.2cm}

%===============================================================================
%	CONTENT
%===============================================================================
% \cvsubsection{Computer Science and Engineering}
\begin{cventries}
%---------------------------------------------------------
\cventry
        {Researcher} 
        {Prisma Health Cancer Institute} 
        {Jul 2022 -- Present} 
        {South Carolina, USA} 
        {
            \begin{cvitemsfree}
                \item{Designed and deployed a normalized MySQL cancer-profiling database with SQLAlchemy ORM; database now exceeds 10,000 curated patient records.}
                % 
                \item{Built a modular Python ETL pipeline for parsing, cleaning, and loading clinical patient datasets using Pandas and NumPy, improving processing speed and data reliability.}
                % 
                \item{Carried out ML/DL experments to classify patient response across 65 drug combinations. Addressed class imbalance and data misingness, while using a 4-metric evaluation system.}
                % 
                \item{Used SHAP for model interpretability, providing actionable insights to clinicians.}
            \end{cvitemsfree}
        }
%---------------------------------------------------------

    \cventry
        {Graduate Research Assistant} 
        {University of South Carolina} 
        {Jan 2022 -- Dec 2024} 
        {South Carolina, USA} 
        {
            \begin{cvitemsfree}
                \item{Efficiently processed \textbf{over 6.9 billion} mass spectrometry data points from 78 raw instrument files, storing results in MySQL using asynchronous approach to optimize ingestion time.}
                % 
                \item{Built deep CNNs and ResNet for binary classification of cancerous tissue using the MS data.}
                % 
                \item{Applied data reduction techniques to 77,000+ small molecule descriptors for unsupervised clustering. Built an ML model for retention time prediction.}  % and chemical pattern discovery
                % 
                \item{Co-developed TransONet: a ResNet-Transformer-UNet hybrid approch, achieving 93.5\% Dice score for aortic segmentation in CT scans.}
            \end{cvitemsfree}
        }

%---------------------------------------------------------

    \cventry
        {Selected Projects} 
        {University of South Carolina} 
        {} 
        {} 
        {
            \begin{cvitemsfree}
                \item{Led 5-member team to build a cloud-native certificate management system using Azure (SQL, Service Bus, Logic Apps, Functions, Data Factory) and Azure DevOps. Used .NET Core 7, EF Core, and React; integrated CI/CD and serverless architecture.}
                % 
                \item{prototyped a RAG-based chatbot for USC Hackathon using LangChain, LangSmith, and MongoDB Atlas Vector DB.}
                % 
                \item{Implemented and modularized RAG-based agent for question answering over research paper datasets, covering prompt versioning, document reasoning, vector stores, and semantic similarity via embeddings.}
                % 
                \item{Designed and deployed a Hybrid GraphRAG Agent using LangGraph, LangSmith, Neo4j, and vision LLMs. Extracted structured data from PDFs (graphs/tables/images) using PyMuPDF4LLM; implemented advanced semantic chunking and document ingestion with progress tracking.}
                % 
                \item{Developed and deployed a semantic segmentation model on AWS SageMaker, trained to segment images into foreground, background, and transition classes.}
                % 
                \item{Completed a project-based PyTorch course focused on computer vision. Implemented binary and multi-class classification, object detection with bounding boxes, and semantic/instance/panoptic segmentation. Developed GANs for image synthesis and evaluated model performance.}
                % 
                \item{Studied the lifecycle of LLMs in a course on Generative AI, covering architecture, fine-tuning, evaluation metrics, and deployment. Applied scaling laws and constraint-aware optimization strategies in LLM projects.}
                % 
                \item{Attended a 5-day Gen AI intensive program by Google covering: foundational models and prompt engineering, embeddings and vector search, generative agents, domain-specific LLMs (e.g., Med-PaLM), and MLOps for generative AI. Hands-on coding throughout using Gemini and Vertex AI.}
                % 
                \item{Implemented Vision Transformer (ViT) models in PyTorch Lightning for image classification tasks.}
                % 
                \item{Fine-tuned Hugging Face ALBERT on TripAdvisor sentiment analysis; handled tokenization, training, and evaluation.}
            \end{cvitemsfree}
        }
%---------------------------------------------------------
    \ifdefstring{\solution}{mschem}{
        \cventry
    {Graduate Research Assistant (Chemical Eng)} % Job title
    {Chemistry and Chemical Engineering Research Center of Iran} % Organization
    {Jan 16 - May 20} % Date(s)
    {Tehran, Iran} % Location
    {
        \begin{cvitemsfree} % Description(s) of tasks/responsibilities
            \item{CFD Simulation of a pilot-scale Bubble column reactor: Hydrodynamics, Temperature, and Concentration}
                % \SubItemN{Collaborated research to numerically simulate a pilot-scale bubble column reactor}
                \SubItemN{Simulation results were compared with real data to inform manufacturing decisions}
                % \medskip
            %---------------------------------------------------------
            \item{CFD simulation coupled with Population Balance Equation (PBE) approach to investigate the impact of process parameters on Crystal Size Distribution (CSD) and hydrodynamics during the crystallization process in a pilot Draft Tube Baffle (DTB) crystallizer}
                % \SubItemN{Attempted a coupled Computational Fluid Dynamics (CFD) and Population Balance Equation (PBE) approach to investigate the impact of process parameters on Crystal Size Distribution (CSD) and hydrodynamics during the crystallization process in a pilot Draft Tube Baffle (DTB) crystallizer}
                \SubItemN{Used experimental design and data analysis (DOE) to acquire data and validate CFD-PBE simulation results}
                % \medskip
            %---------------------------------------------------------
            \item{Formulated and characterized the viscosity, strain, and storage modulus of a polymer composite, using experimental design (DOE), to achieve low viscosity final product and improving the production process}
                % \SubItemN{Used experimental design to formulate and characterize the viscosity, strain, and storage modulus of a polymer composite and their interactions}
                \SubItemN{Improved the production process by optimizing the rheological properties of the composite and achieving a low viscosity final composite}
            %---------------------------------------------------------
            % \bigskip \medskip
        \end{cvitemsfree}
    }
    }{}
%---------------------------------------------------------
    % \cventry
    % {Finance Investor} % Job title
    % {} % Organization
    % {2015 - 2020} % Date(s)
    % {} % Location
    % {
    %     \SubItem{\vspace{-0.4cm}Experienced a 150\% growth rate, in 5 years, as a result of utilizing money management techniques and applying both fundamental and technical analysis to Iranian stocks.
    %     }
    % }
%---------------------------------------------------------
\end{cventries}