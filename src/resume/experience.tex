%-------------------------------------------------------------------------------
%	SECTION TITLE
%-------------------------------------------------------------------------------

\cvsection{Experience}
\vspace*{-0.2cm}

%===============================================================================
%	CONTENT
%===============================================================================
% \cvsubsection{Computer Science and Engineering}
\begin{cventries}
%---------------------------------------------------------
    \cventry
        {Graduate Research Assistant} % Job title
        {University of South Carolina} % Organization
        {Jan 22 - Dec 24}
        {SC, US} % Location
        {
            \begin{cvitemsfree} % Description(s) of tasks/responsibilities
                \item{Managed high-throughput liquid chromatography-mass spectrometry (LCMS) data (\textbf{over 6.9 billion data points} across 78 files), storing it into a MySQL database by leveraging the advantages of SQLAlchemy's asyncio methods to expedite the process. Explored neural networks like ResNet and CNN to differentiate between cancerous and surrounding tissues based on the processed data}
                %
                \item{Applied dimension reduction techniques (PCA, t-SNE, ISOMAP) to perform cluster analysis and reveal natural groupings in data comprising over \textbf{77,000} small molecules and their chemical descriptors. Performed retention time prediction for these small molecules based on their chemical descriptors. Trained baseline models for initial feature selection and data understanding.}
                %
                \item{Designed a Cancer-Profiling Database adhering to normal form standards, and utilized SQLAlchemy's Object Relational Mapping (ORM) for MySQL connectivity. The database currently holds over 10,000 records and continues to grow}
                %
                \item{Developed a Python pipeline for parsing, cleaning, verifying, and loading data onto the database, leveraging multiple frameworks including Numpy, Pandas, etc. The data are processed from Prisma Health, enhancing the efficiency and accuracy of data management processes}
                %
                \item{Conducted research on machine learning (ML) algorithms (using the Scikit-Learn library), including tree-based methods (Decision Trees, Random Forest), and gradient boosting techniques (XGBoost, AdaBoost, CatBoost), to classify patient responses to different combinations of 65 medications. Employed evaluation metrics (CM, F1, recall, precision, etc.) to assess model performance. Visualized the result (using SHAP) for enhanced interpretability and decision-making, while addressing class imbalance issues (using down sampling, SMOTE) and ensuring robust and reliable predictions.}
                %
                \item{Contributed to the development of TransONet, a deep learning model with ResNet encoders and Transformer bridges integrated into a UNet structure, achieving \textbf{93.5\% Dice accuracy} for aortic segmentation in CT scans.}
                %
                \item{Led a \textbf{5-member team} in developing an End‑to‑End Web App (Certificate Management System) in an Azure Cloud Native environment. Leveraging various Azure services including SQL Database, Service Bus, Function Apps (with Blob Storage and Service Bus triggers), Logic Apps (with Microsoft 365 Outlook connector), and Data Factory. Applied agile methodologies and CI\/CD practices throughout the development process. Implementing features such as data integration from external feeds, CRUD operations with notifications, dashboard, user profile, and certificate catalog. Used Dotnet (.NET) Entity Framework Core, .NET Core 7.0 (C\#), REST API, and React, in addition to following a microservice architecture with serverless components.}
                %
                \item{Classified images using Vision Transformer (ViT) architectures using PyTorch and Lightning following object orientation programming (OOP).}
                %
                \item{Conducted sentiment analysis on TripAdvisor reviews dataset using Hugging Face ALBERT model, including text tokenization, and fine-tuning.}
            \end{cvitemsfree}
        }
%---------------------------------------------------------
    \ifdefstring{\solution}{mschem}{
        \cventry
    {Graduate Research Assistant (Chemical Eng)} % Job title
    {Chemistry and Chemical Engineering Research Center of Iran} % Organization
    {Jan 16 - May 20} % Date(s)
    {Tehran, Iran} % Location
    {
        \begin{cvitemsfree} % Description(s) of tasks/responsibilities
            \item{CFD Simulation of a pilot-scale Bubble column reactor: Hydrodynamics, Temperature, and Concentration}
                % \SubItemN{Collaborated research to numerically simulate a pilot-scale bubble column reactor}
                \SubItemN{Simulation results were compared with real data to inform manufacturing decisions}
                % \medskip
            %---------------------------------------------------------
            \item{CFD simulation coupled with Population Balance Equation (PBE) approach to investigate the impact of process parameters on Crystal Size Distribution (CSD) and hydrodynamics during the crystallization process in a pilot Draft Tube Baffle (DTB) crystallizer}
                % \SubItemN{Attempted a coupled Computational Fluid Dynamics (CFD) and Population Balance Equation (PBE) approach to investigate the impact of process parameters on Crystal Size Distribution (CSD) and hydrodynamics during the crystallization process in a pilot Draft Tube Baffle (DTB) crystallizer}
                \SubItemN{Used experimental design and data analysis (DOE) to acquire data and validate CFD-PBE simulation results}
                % \medskip
            %---------------------------------------------------------
            \item{Formulated and characterized the viscosity, strain, and storage modulus of a polymer composite, using experimental design (DOE), to achieve low viscosity final product and improving the production process}
                % \SubItemN{Used experimental design to formulate and characterize the viscosity, strain, and storage modulus of a polymer composite and their interactions}
                \SubItemN{Improved the production process by optimizing the rheological properties of the composite and achieving a low viscosity final composite}
            %---------------------------------------------------------
            % \bigskip \medskip
        \end{cvitemsfree}
    }
    }{}
%---------------------------------------------------------
    % \cventry
    %     {} % Organization
    %     {Finance Investor} % Job title
    %     {2015 - 2020} % Date(s)
    %     {} % Location
    %     {
    %         \SubItem{\vspace{-0.4cm}Experienced a 150\% growth rate, in 5 years, as a result of utilizing money management techniques and applying both fundamental and technical analysis to Iranian stocks.
    %         }
    %     }
%---------------------------------------------------------
\end{cventries}